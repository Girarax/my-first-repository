% GIRARDOT AXEL COBOTIQUE 5A

\documentclass[english]{article}
\usepackage[a4paper, total={18cm, 26cm}]{geometry}
\usepackage{graphicx}
\usepackage{amsmath}
\usepackage{amssymb}
\usepackage{booktabs}
\usepackage{nicefrac}
\usepackage{algorithm}
\usepackage[algo2e]{algorithm2e} 
\usepackage{multirow}
\usepackage{mwe}
\usepackage{float}
\usepackage{ulem}
\usepackage{indentfirst}

\usepackage[pagebackref,breaklinks,colorlinks]{hyperref}

\usepackage{balance}
\usepackage{comment}
\usepackage{fancyhdr}

\usepackage[capitalize]{cleveref}
\crefname{section}{Sec.}{Secs.}
\crefname{table}{Tab.}{Tabs.}
\Crefname{section}{Section}{Sections}
\Crefname{table}{Table}{Tables}

\newcommand{\modified}[1]{\color{black}{{#1 }}\color{black}}
\newcommand{\etal}{\MakeLowercase{\textit{et al.}}}
\newcommand{\atan}{\arctan}
\newcommand{\plusbinomial}[3][2]{(#2 + #3)^#1}

\usepackage{multicol}

% \IEEEoverridecommandlockouts
\begin{document}

\title{Simultaneous Localization And Mapping (SLAM)}

\author{
Andreas Birk and Max Pfingsthorn
\thanks{The authors are with the Dept. of Electrical Engineering and Computer Science, Jacobs University Bremen, 28751 Bremen, Germany.
\textbf{a.birk@jacobs-university.de}}
}

\nocite{*}
\maketitle

%\begin{multicols}{2}
%%%%%%%%% ABSTRACT
\begin{abstract}
    This article gives an overview introduction to Simultaneous Localization And Mapping (SLAM), i.e., probabilistic
    methods to generate a 2D or 3D map of unknown areas under imperfect localization. The article provides a
    survey of the theoretical basis of SLAM as well as the core background information about the underlying techniques
    for implementing actual SLAM systems
\end{abstract}

\section{Introduction}
\noindent

\section{SLAM Frontend}
\subsection{Egomotion Estimation with Internal Sensors and Models}
\noindent

\section{Experimental results}


{\small
\bibliographystyle{IEEEtran}
\bibliography{references_exemple}
}
\end{document}